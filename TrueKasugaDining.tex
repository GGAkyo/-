\documentclass[
paper=a6,
fontsize=7pt,jafontsize=7pt,
baselineskip=1.6zh,
book,
hanging_punctuation,
head_space=10truemm,foot_space=10truemm
]{jlreq}
\usepackage{pxrubrica}
\usepackage{multicol}
\usepackage{luatexja-fontspec}
\setmainjfont[
]{YuMincho}
\usepackage{lltjext}
\setlength{\parindent}{0pt}
{\catcode`\^^M=\active%
\gdef\xobeylines{\catcode`\^^M\active \def^^M{\par\leavevmode}}%
\global\def^^M{\par\leavevmode}}
\usepackage[hyphens]{url}

\NewPageStyle{honbun}{% 本文のページスタイル
running_head_position=top-left,% 柱の位置
running_head_font=\footnotesize,% 柱のフォントサイズ
nombre_position=bottom-left,% ノンブルの位置
odd_running_head=_chapter,% 奇数ページの柱
even_running_head={真・春日食堂について},% 偶数ページの柱
}

\begin{document}
\title{真・春日食堂について}
\date{}
\author{幻月盈虚}
\maketitle

\pagestyle{honbun}
\xobeylines
\chapter{はじめに}

 この記事は、mastアドカレ12日目の記事です。
 \url{https://adventar.org/calendars/10425}
 前回11日目の記事は、粗茶ですがさんの『映画を映画館で観るようにしてみた』の記事となっています。
 \url{https://note.com/green_tea_725/n/n1a4108042a09}
 
 さて、さっそく本題に入りましょう。
 
 みなさんは『真・春日食堂』という意味不明な文字を見たことが、あるいは聞いたことがありますでしょうか。
 もしくは『日本国際学園大学』や『筑波学院大学』といえば一度は目にしたことがあるのではないでしょうか。
 
 先に解説を行うと、真・春日食堂という施設は当然のことながら存在しません。日本国際学園大学(旧筑波学院大学)の食堂を勝手にそう呼んでいるだけです。
 営業時間は特別な行事がない平日の11:30から14:00(ただしラストオーダーは13:30)
 メニューは日替わり丼・日替わり定食 ともに500円
 もつ煮込み定食(最近は牛スジ煮込み)400円
 カレー(タイミングによって種類が変わる)400円
 うどん・そば ともに350円
 ごはん 100円
 麺大盛り2倍 100円
 となっております。
 
 場所は、日本国際学園大学と書かれている、何故か開き戸タイプの珍しい自動ドアから入り、左側にまっすぐ歩いていった突き当りの第2食堂と書かれているところです。ちなみに第1食堂は(多分)ありません。手前エリアが1棟、廊下を歩いていった奥側が2棟となっていて、2棟にあるから第2食堂なのではないかと思われます。
 
 春日エリアにある筑波大学本来の食堂は、コロナ禍による規模の縮小や物価高騰などで、他エリアの食堂などと比べても、悲しい(マイルド表現)ことになってしまっています。そこで、東京の大学生とかはよくやるらしい、他大学の学食を食べに行く、というのをやってみたのが我々の中での真春日ブームの始まりです。
 真春日は、前述の通り安いというのはもちろん、提供がものすごく早い、量が多いなどの素晴らしい点もあります。量がどのくらい多いかというと、調子の良い日は米が1合ぐらい盛られて出てきます。うどんとかもたぶん1玉あります。でっかい肉塊が3きれ出てきたこともあります。
 ぜひ、春日エリアに訪れた際は、一度お立ち寄りください。
 
\chapter{真春日の歴史}
 先ほど『我々の中でのブーム』という表現を使ったとおり、私は別に考案者でも先駆者でもありません。そもそもここの学食が美味しいと教えてくれたのはEkasilicon先輩という22生の方(しかも別に学類は春日エリアじゃない)ですし、ツイッターで検索してみるとかなり古い文献がヒットします。
 
 学院大の食堂のことを真の春日食堂と呼んだのは多分この人が初出
 \url{https://x.com/waffle_mk353/status/146295401107030017}
 
 春日ラウンジのことを真春日と呼ぶ派閥も一定数いる
 \url{https://x.com/kiyu_tomo1059/status/401710210923765760}
 \url{https://x.com/Taiyaki03R226/status/459557153292816384}
 \url{https://x.com/mocachimoca/status/1724695411645153736}
 
 今が一体第何次ブームなのかは分かりませんが、少なくとも10年以上前にはこうして、真の春日食堂で学食を食べるという行為は行われていたようです。
 
 また、日本国際学園大学のGoogleの口コミを見てみると、面白いコメントがありました。
 \url{https://g.co/kgs/7NJxHnZ}
 『学食がとても美味しいです。赤字覚悟の値段設定で本格フランス料理を提供しています。』
 とのことで、5年前にはフランス料理が提供されていた? ようです。
 \url{https://g.co/kgs/NtcNq2o}
 あと8年前には猫もいたらしい。
 
 フランス料理が提供されていたということでインターネットアーカイブで色々と探してはみたのですが、情報の真偽は確かめることが出来ませんでした。これは日本国際学園大学の人に聞かないと確かなことはいえなさそうです。
 その代わり、昔はちょっと雰囲気違ったっぽいことは分かりました。
 \url{https://web.archive.org/web/20150529062556/http://www.tsukuba-g.ac.jp/intro/establishment/}
 (これリンク飛べないですね。webアーカイブってどうやってリンク貼れば良いんだろ)

 ここらで現在に目を向けてみると、公式アカウントがあるのも分かります。
 
 こちらは、学食の公式インスタグラムアカウント
 \url{https://www.instagram.com/tsukubauniv.shokudo/?hl=af}
 こちらは大学の公式TikTokアカウント
 \url{https://www.tiktok.com/@jiu2024/video/7316736310226586887}
 
 インスタの方は特にフォローしておくべきで、日替わり丼や日替わり定食の情報を知ることが出来ます。(なお、開店時間は11時30分なのですが、開店時間前にメニューが投稿されることは稀です。行ったほうが早いことが多い)

\chapter{表記揺れについて}
 この文章内では、『真・春日食堂』あるいは『真春日』と表記するようにしているのですが、口伝による継承の結果、『シンカスガ』という響きに漢字を当てた、『新春日』という表記も使われています。あとは、エヴァとかゴジラみたいに、『シン春日』という人もいます。
 また、ごくごく一部ですが、単に『春日食堂』とだけ言って真春日のことを指している人もいるため、注意が必要です。
 春日食堂に日替わりメニューがあると勘違いして行った人の図
 \url{https://x.com/okakakkka/status/1805458155260067899}
 
 この口コミによる広まり方で面白いのが、筑波大学の春日エリアにある学食の呼び方の変容です。
 レトロニムという、新しいものと区別するために既存のものを区別するために作られた語があります。例としては、特急が出来たことにより『各駅停車』という言葉が生まれたり、カラーテレビが出来たことにより『白黒テレビ』という言葉が生まれたりしたものが挙げられます。
 ダイヤモンド M-1決勝ネタ『レトロニム』/M-1グランプリ2022 とか
 \url{https://youtu.be/cTu3n8TGwFA?si=qyoRxp_HNtOI1p6A}
 寺田寛明『未来のレトロニム』 とか
 \url{https://youtu.be/tcy6dR3KLwU?si=t4NBedHSGfN2U0vE}
 を見ると分かりやすいと思います。
 
 この結果、まだ用例は少ないものの、『真春日』という言葉の表記揺れである『新春日』に対するレトロニムが起こり、従来の春日食堂のことを『旧春日』と呼ぶようになってきています。もし数年後にもこの、旧春日という言葉だけ残っていた場合意味が分からなくて大変な事になってしまうと思うので、ここに記録を残しておきたいと思います。
 
\chapter{アンケート取ってみた}
 この記事を書くにあたり、
 ・メ創23生のDiscordサーバー
 ・げんしけんのDiscordサーバー
 ・Twitter
 で、簡単なアンケートを行いました。現在までに55件もの回答をいただいております。ご協力、本当にありがとうございました。
 設問は、「何生か」「学類はどこか」「真春日を使ったことがあるか」の3つとなっています。普通に母集団は偏りまくっているので割合に信頼性とかはないですが、絶対的な数においてはある程度信頼できると思われます。
 
 ではまず、私の周り、メ創23生の人たちの回答を見てみましょう。
 利用したことがあると回答した数、14件中13件。
 そもそも使ったことがない人は回答しない傾向があるとかはともかく、確実に13人は行ったことがあるらしい。体感、春日で授業があるとき、週1以上で使うのは20人くらい、行ったことがあるのは30人ぐらいだと思う。(メ創は1学年50人ほど)。
 じゃあ、他の学年のメ創は──?
 20生(M1):2件中2件
 21生:1件中0件
 22生:2件中2件
 24生:3件中2件
 回答数が少ないからなんとも言えんけど、なんか院生にも広まってる。
 
 じゃ、じゃあ、知識情報・図書館学類も見てみようか──。
 22生:4件中2件
 23生:4件中3件
 24生:3件中3件
 春日の民には、上から下まで浸透しつつあるらしい……。
 
 他の利用したことがある人たちも見てみるか──。
 23生 医学類:1件
 23生 応用理工学類:1件
 23生 芸術専門学群:1件
 23生 地球学類:1件
 24生 情報科学類:1件
 24生 生物資源学類:1件
 24生 総合学域軍:1件
 なんか軍に所属してるやつがいたけど、春日じゃない他学類にも23生を中心に広まりつつあるっぽい。てか情科とか一番対極みたいな位置じゃないか?
 
 最後に、行ったことないと回答してくれた人も書いておこう。
 20生	社会工学類
 21生	国際総合学類
 22生	生物資源学類
 22生	物理学類
 23生	医療科学類
 23生	応用理工学類
 23生	社会工学類
 23生	物理
 24生	Coins(情報科学類のこと)
 24生	芸術専門学群
 24生	総合2類
 24生	総合学域群第3類
 24生	総合学域群第一類
 24生	総合学域群第一類
 24生	物理学類
 割と幅広い学類・学年に回答してもらえている。(もちろんサンプル数としては不十分だけども)
 合計で55件の回答のうち、利用したことがあると回答した数は、34件だった。
 少なくとも34人は、日本国際学園大学の学食を食べたらしい。
 というかこのアンケート、広まっていると言っても知り合い程度にしか回答してもらえていないので、何人か透けて見えるのが面白い。
 せっかくなら、旧春日の利用とともに回答を取ったら面白かったかもしれない。
 
 この8ヶ月ほどでここまで急速に広まっているので、春日の25生とかはもはや普通に真春日に行くようになっているかもしれない。
 こんなことを書いているうちに、推薦入試で合格した人が発表された。おめでとう。

\chapter{最後に}
 日本国際学園大学の時間割は、筑波大学とかなり異なるものとなっています。
 具体的にいうと、我々が2限終わりに昼ご飯を食べに行くときには、向こうは授業中です。特に、食堂の向かいの教室では講義が行われていることがあります。迷惑にならないように、廊下などでは騒ぎすぎないようにしましょう。
 
 あと、Ekasiliconさんに名前を出して良いか許可取ったときのやり取りだけ載せておきます。
 幻月「アドカレで真・春日食堂の記事を書こうとしているのですが、教えてくれた先輩として名前紹介しても良いでしょうか?」
 Eka「草 いいよ」
 幻月「ありがとうございます!」
 Eka「Prefix-prefix Suffix-Suffixをよろしく…」
 (意味が分からなくて数秒フリーズ)
 Eka「日本 国際:大学の前につきがち 学園 大学:後ろにつきがち」
 幻月「草」
\end{document}